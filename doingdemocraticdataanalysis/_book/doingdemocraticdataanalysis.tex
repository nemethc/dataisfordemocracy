\documentclass[]{book}
\usepackage{lmodern}
\usepackage{amssymb,amsmath}
\usepackage{ifxetex,ifluatex}
\usepackage{fixltx2e} % provides \textsubscript
\ifnum 0\ifxetex 1\fi\ifluatex 1\fi=0 % if pdftex
  \usepackage[T1]{fontenc}
  \usepackage[utf8]{inputenc}
\else % if luatex or xelatex
  \ifxetex
    \usepackage{mathspec}
  \else
    \usepackage{fontspec}
  \fi
  \defaultfontfeatures{Ligatures=TeX,Scale=MatchLowercase}
\fi
% use upquote if available, for straight quotes in verbatim environments
\IfFileExists{upquote.sty}{\usepackage{upquote}}{}
% use microtype if available
\IfFileExists{microtype.sty}{%
\usepackage[]{microtype}
\UseMicrotypeSet[protrusion]{basicmath} % disable protrusion for tt fonts
}{}
\PassOptionsToPackage{hyphens}{url} % url is loaded by hyperref
\usepackage[unicode=true]{hyperref}
\hypersetup{
            pdftitle={Doing Democratic Data Analysis},
            pdfauthor={Corban Nemeth},
            pdfborder={0 0 0},
            breaklinks=true}
\urlstyle{same}  % don't use monospace font for urls
\usepackage{natbib}
\bibliographystyle{apalike}
\usepackage{color}
\usepackage{fancyvrb}
\newcommand{\VerbBar}{|}
\newcommand{\VERB}{\Verb[commandchars=\\\{\}]}
\DefineVerbatimEnvironment{Highlighting}{Verbatim}{commandchars=\\\{\}}
% Add ',fontsize=\small' for more characters per line
\usepackage{framed}
\definecolor{shadecolor}{RGB}{248,248,248}
\newenvironment{Shaded}{\begin{snugshade}}{\end{snugshade}}
\newcommand{\KeywordTok}[1]{\textcolor[rgb]{0.13,0.29,0.53}{\textbf{#1}}}
\newcommand{\DataTypeTok}[1]{\textcolor[rgb]{0.13,0.29,0.53}{#1}}
\newcommand{\DecValTok}[1]{\textcolor[rgb]{0.00,0.00,0.81}{#1}}
\newcommand{\BaseNTok}[1]{\textcolor[rgb]{0.00,0.00,0.81}{#1}}
\newcommand{\FloatTok}[1]{\textcolor[rgb]{0.00,0.00,0.81}{#1}}
\newcommand{\ConstantTok}[1]{\textcolor[rgb]{0.00,0.00,0.00}{#1}}
\newcommand{\CharTok}[1]{\textcolor[rgb]{0.31,0.60,0.02}{#1}}
\newcommand{\SpecialCharTok}[1]{\textcolor[rgb]{0.00,0.00,0.00}{#1}}
\newcommand{\StringTok}[1]{\textcolor[rgb]{0.31,0.60,0.02}{#1}}
\newcommand{\VerbatimStringTok}[1]{\textcolor[rgb]{0.31,0.60,0.02}{#1}}
\newcommand{\SpecialStringTok}[1]{\textcolor[rgb]{0.31,0.60,0.02}{#1}}
\newcommand{\ImportTok}[1]{#1}
\newcommand{\CommentTok}[1]{\textcolor[rgb]{0.56,0.35,0.01}{\textit{#1}}}
\newcommand{\DocumentationTok}[1]{\textcolor[rgb]{0.56,0.35,0.01}{\textbf{\textit{#1}}}}
\newcommand{\AnnotationTok}[1]{\textcolor[rgb]{0.56,0.35,0.01}{\textbf{\textit{#1}}}}
\newcommand{\CommentVarTok}[1]{\textcolor[rgb]{0.56,0.35,0.01}{\textbf{\textit{#1}}}}
\newcommand{\OtherTok}[1]{\textcolor[rgb]{0.56,0.35,0.01}{#1}}
\newcommand{\FunctionTok}[1]{\textcolor[rgb]{0.00,0.00,0.00}{#1}}
\newcommand{\VariableTok}[1]{\textcolor[rgb]{0.00,0.00,0.00}{#1}}
\newcommand{\ControlFlowTok}[1]{\textcolor[rgb]{0.13,0.29,0.53}{\textbf{#1}}}
\newcommand{\OperatorTok}[1]{\textcolor[rgb]{0.81,0.36,0.00}{\textbf{#1}}}
\newcommand{\BuiltInTok}[1]{#1}
\newcommand{\ExtensionTok}[1]{#1}
\newcommand{\PreprocessorTok}[1]{\textcolor[rgb]{0.56,0.35,0.01}{\textit{#1}}}
\newcommand{\AttributeTok}[1]{\textcolor[rgb]{0.77,0.63,0.00}{#1}}
\newcommand{\RegionMarkerTok}[1]{#1}
\newcommand{\InformationTok}[1]{\textcolor[rgb]{0.56,0.35,0.01}{\textbf{\textit{#1}}}}
\newcommand{\WarningTok}[1]{\textcolor[rgb]{0.56,0.35,0.01}{\textbf{\textit{#1}}}}
\newcommand{\AlertTok}[1]{\textcolor[rgb]{0.94,0.16,0.16}{#1}}
\newcommand{\ErrorTok}[1]{\textcolor[rgb]{0.64,0.00,0.00}{\textbf{#1}}}
\newcommand{\NormalTok}[1]{#1}
\usepackage{longtable,booktabs}
% Fix footnotes in tables (requires footnote package)
\IfFileExists{footnote.sty}{\usepackage{footnote}\makesavenoteenv{long table}}{}
\usepackage{graphicx,grffile}
\makeatletter
\def\maxwidth{\ifdim\Gin@nat@width>\linewidth\linewidth\else\Gin@nat@width\fi}
\def\maxheight{\ifdim\Gin@nat@height>\textheight\textheight\else\Gin@nat@height\fi}
\makeatother
% Scale images if necessary, so that they will not overflow the page
% margins by default, and it is still possible to overwrite the defaults
% using explicit options in \includegraphics[width, height, ...]{}
\setkeys{Gin}{width=\maxwidth,height=\maxheight,keepaspectratio}
\IfFileExists{parskip.sty}{%
\usepackage{parskip}
}{% else
\setlength{\parindent}{0pt}
\setlength{\parskip}{6pt plus 2pt minus 1pt}
}
\setlength{\emergencystretch}{3em}  % prevent overfull lines
\providecommand{\tightlist}{%
  \setlength{\itemsep}{0pt}\setlength{\parskip}{0pt}}
\setcounter{secnumdepth}{5}
% Redefines (sub)paragraphs to behave more like sections
\ifx\paragraph\undefined\else
\let\oldparagraph\paragraph
\renewcommand{\paragraph}[1]{\oldparagraph{#1}\mbox{}}
\fi
\ifx\subparagraph\undefined\else
\let\oldsubparagraph\subparagraph
\renewcommand{\subparagraph}[1]{\oldsubparagraph{#1}\mbox{}}
\fi

% set default figure placement to htbp
\makeatletter
\def\fps@figure{htbp}
\makeatother

\usepackage{booktabs}

\title{Doing Democratic Data Analysis}
\author{Corban Nemeth}
\date{2020-04-15}

\begin{document}
\maketitle

{
\setcounter{tocdepth}{1}
\tableofcontents
}
\chapter*{Preface}\label{preface}
\addcontentsline{toc}{chapter}{Preface}

I believe that data, \emph{in the hands of public administrators and
policy analysts}\footnote{\emph{Not IT departments}}, has the power to
transform the way government works.

Big questions will, and should, be asked of big data--- the role of
government in regulating algorithmic bias, facial recognition, and
consumer data privacy is a vital conversation. However, these topics
should not detract or deter public administrators and policy analysts
from leaning into \textbf{small data} for decision-making purposes.

Public administrators and analysts who are data literate will be able to
make and inform better decisions while avoiding the pitfalls posed by
the latest technological trends. This book represents an opportunity for
public administrators and policy analysts to join their subject matter
expertise with foundation principles and practices of democratic data
analysis--- data analysis that is \textbf{transparent, relevant, and
grounded in the context of ethical and effective governance}.

\section*{Who is this guide for?}\label{who-is-this-guide-for}
\addcontentsline{toc}{section}{Who is this guide for?}

This guide is for:

\begin{itemize}
\tightlist
\item
  the budget analyst at the Department of Fish and Wildlife who has to
  compile a monthly report analyzing revenues,
\item
  the manager at the Department of Social and Health Services who is
  tracking inventory, and
\item
  the research analyst working for the state Legislature who wants to
  incorporate data into her work session on the latest policy debate.
\end{itemize}

\section*{What will I learn?}\label{what-will-i-learn}
\addcontentsline{toc}{section}{What will I learn?}

You will learn an opinionated framework for data analysis in public
sector organizations. By opinionated, I mean that I will teach you what
I think is the right way to do things given my own experience as a
public sector policy and data analyst. Your experience might differ--
and that's great. I hope that where you can use your experience in place
of mine, you do to the fullest extent. With that in mind, it is often
said that you have to know the rules to break them, so I will teach you
the ``rules'' as I understand them.

\begin{Shaded}
\begin{Highlighting}[]
\KeywordTok{summary}\NormalTok{(cars)}
\end{Highlighting}
\end{Shaded}

\begin{verbatim}
##      speed           dist       
##  Min.   : 4.0   Min.   :  2.00  
##  1st Qu.:12.0   1st Qu.: 26.00  
##  Median :15.0   Median : 36.00  
##  Mean   :15.4   Mean   : 42.98  
##  3rd Qu.:19.0   3rd Qu.: 56.00  
##  Max.   :25.0   Max.   :120.00
\end{verbatim}

\chapter{Introduction}\label{intro}

\section{WRITE ABOUT BIG THEMES FIRST AND
HERE}\label{write-about-big-themes-first-and-here}

\section{DEFINE PRINCIPLES AND
PRACTICES}\label{define-principles-and-practices}

\section{What is democratic data
analysis?}\label{what-is-democratic-data-analysis}

This is little-d democracy here, people. Not a guide to data analysis
for Democrats, but an approach to data that empowers the subject matter
experts within public organizations to do their best work.

\section{Do I need to be an Excel god before I
begin?}\label{do-i-need-to-be-an-excel-god-before-i-begin}

\textbf{No.}

\section{What tools should I use?}\label{what-tools-should-i-use}

This guide is written to be technology-agnostic and is not a tutorial on
specific tools. The changing nature of technology and the lag of state
agencies to pick up tools means that the principles outlined in this
book supersede specific technologies.

However, you should start somewhere. This guide will include examples in
both Excel and \texttt{R}. Government runs on Excel, so all of the
examples and exercises will be Excel compatible. If you are comfortable
with Excel\footnote{aka you use \texttt{vlookup}, \texttt{index(match)},
  pivot tables, or \emph{Get \& Transform} on a somewhat regular basis}
and want to challenge yourself, boost your resume, and become a data
wizard\footnote{For example, I used R to create this entire website}, I
would highly recommend learning \texttt{R}. This guide will wshow
examples on how to analyze data democratically using both tools, but
focusing on the principles and practices that are vital regardless of
technology. I will link to specific resources that provide more detailed
walk through's as necessary.

\section{How to use this book}\label{how-to-use-this-book}

\section{Quickstart Guide to Excel}\label{quickstart-guide-to-excel}

Get \& Transform and Data Model

\section{Quickstart Guide to R}\label{quickstart-guide-to-r}

I would \textbf{HIGHLY} recommend you take an afternoon to work through
Chapters 1, 2, and 3 of
``\href{https://rstudio-conf-2020.github.io/r-for-excel/index.html}{\emph{R
for Excel Users}}''. You will thank me later.

\section{How to Think about Democratic Data
Analysis}\label{how-to-think-about-democratic-data-analysis}

What is data analysis? It may be easier to start with what data analysis
isn't.

\begin{enumerate}
\def\labelenumi{\arabic{enumi}.}
\tightlist
\item
  Data analysis isn't math in Excel or a set of specific calculations.
\end{enumerate}

\begin{itemize}
\tightlist
\item
  Calculations are great, but \texttt{a7\ +\ b8} in Excel is
  deterministic. It gives you one answer. This book is not interested in
  data analysis that gives you the right answer, because there is no
  such thing. There are many answers to many questions, depending on how
  those questions are asked and how the data is analyzed.
\end{itemize}

\begin{enumerate}
\def\labelenumi{\arabic{enumi}.}
\setcounter{enumi}{1}
\tightlist
\item
  Data analysis isn't statistics.
\end{enumerate}

\begin{itemize}
\tightlist
\item
  Look, regressions are great, but that isn't the crux of this book.
  This book is about reading and telling the story of your data in a way
  that can complement expertise and experience to make better decisions.
  Statistics are often used as a cheap stand-in for domain expertise and
  are often abused in favor of trusting the analyst or administrator to
  back up their assumptions with both quantitative and qualitiative
  data.
\end{itemize}

\begin{enumerate}
\def\labelenumi{\arabic{enumi}.}
\setcounter{enumi}{2}
\tightlist
\item
  Data analysis isn't research methods.
\end{enumerate}

\begin{itemize}
\tightlist
\item
  No set of tools and practices can stand in for asking the right
  questions, and transforming data into information to answer that
  question. This book will give you the tools to work with your
  quantitative data to answer relevant questions, but all good analysis
  begins with a good question.
\end{itemize}

So what are the principles of democratic data analysis?

\begin{enumerate}
\def\labelenumi{\arabic{enumi}.}
\tightlist
\item
  Democratic data analysis gives power to the subject matter expert.
\end{enumerate}

\begin{itemize}
\tightlist
\item
  Working with your own data lets you answer your question, rather than
  outsource those questions to IT departments who don't have the context
  necessary to turn data into information.
\end{itemize}

\begin{enumerate}
\def\labelenumi{\arabic{enumi}.}
\setcounter{enumi}{1}
\tightlist
\item
  Democratic data analysis is easily shared and reproduced.
\item
  Democratic data analysis is honest about its assumptions and
  limitations.
\item
  Democratic data analysis is structured to be easily understood and
  communicated.
\item
  Democratic data analysis is approached methodically.
\end{enumerate}

\section{Alternate Forms of Data
Analysis}\label{alternate-forms-of-data-analysis}

I want to contrast democratic data analysis with alternate forms of
analysis that I have seen in my time working in the public sector.

\subsection*{Oligarchic Data Analysis}\label{oligarchic-data-analysis}
\addcontentsline{toc}{subsection}{Oligarchic Data Analysis}

\begin{itemize}
\tightlist
\item
  small group of ``data people''
\end{itemize}

\subsection*{Monarchic Data Analysis}\label{monarchic-data-analysis}
\addcontentsline{toc}{subsection}{Monarchic Data Analysis}

\begin{itemize}
\tightlist
\item
  Spreadsheets created decades ago, patched as necessary, and owned by a
  single or small group of people. Very unnaproachable. Can't ask
  questions.
\end{itemize}

\subsection*{Dictitorial Data Analysis}\label{dictitorial-data-analysis}
\addcontentsline{toc}{subsection}{Dictitorial Data Analysis}

*Not reproducible, all hard-coded, can't ask questions,

\subsection*{Technocratic Data
Analysis}\label{technocratic-data-analysis}
\addcontentsline{toc}{subsection}{Technocratic Data Analysis}

*Excel workbooks straight from the depths of hell. Formulas on formulas
on macros.

\section{Principles of Democratic Data
Analysis}\label{principles-of-democratic-data-analysis}

\begin{itemize}
\tightlist
\item
  Think in terms of fields, not values
\item
  Leave breadcrumbs for others (and your future self)
\item
  Create a data pipeline and DO NOT DESTROY UNDERLYING DATA-- build
\item
  Make assumptions, and document them!
\item
  Show AND tell your results
\end{itemize}

\section[The Language of Data Analysis]{\texorpdfstring{The Language of
Data Analysis\footnote{Adapted from Hadley Wickham's paper on
  \href{https://vita.had.co.nz/papers/tidy-data.pdf}{Tidy Data}}}{The Language of Data Analysis}}\label{the-language-of-data-analysis}

If you have ever typed a formula in excel, congrats! You are a
programmer. Embrace it! If you haven't, you will soon, and then you will
be a programmer too. Excel is the world's largest and most underrated
programming language. Now, you just have to embrace thinking
programmaticlly-- tidy. It's like grammar. There are rules so these
sentances (hopefully) make sense to you, the reader. Similarily, by
following common conventions of tidy data analysis, others will be able
to ``read'' your analysis like you are reading this sentance. And also,
like grammar, you can break the rules-- but it helps to know them first.

Here are a couple definitions that will help as you move through this
text. Don't worry about memorizing them, as I will refer back to these
definitions frequently.

\begin{itemize}
\tightlist
\item
  Fields

  \begin{itemize}
  \tightlist
  \item
    A field is a fancy name for a column. From here on out, every
    calculation, manipulation, formula, you name it, will be on a
    column. I want you to forget that you could ever modify a lone cell
    in Excel. No more formulas in cells. No more typing in values to a
    cell. Certainly no more writing over data in a cell. Democratic data
    analysis depends on formulas that work on entire fields. Everything
    you would need to do to a single cell in Excel can-- and should!--
    be done to an entire column. This will be immensly valuable, as you
    will hopefully see while working through this material.
  \end{itemize}
\item
  Variables

  \begin{itemize}
  \tightlist
  \item
    A variable is something in your data that can change. That's it!
    Variables become very important when looking at how to structure
    your data.
  \end{itemize}
\item
  Observations

  \begin{itemize}
  \tightlist
  \item
    Observations make up the rows of your dataset. Each observation
    should correspond to a specific ``thing.'' This will make more sense
    later, I promise.
  \end{itemize}
\item
  Values

  \begin{itemize}
  \tightlist
  \item
    Values are the actual data in your table. Each value belongs to 1
    (one) observation and 1 (one) variable.
  \end{itemize}
\item
  Table

  \begin{itemize}
  \tightlist
  \item
    A table is the grouping of all observations of a similar type.
  \end{itemize}
\end{itemize}

\chapter{Tidy Data}\label{tidy-data}

\section{Cleaning vs Tidying}\label{cleaning-vs-tidying}

My wife gives me a hard time because I hate cleaning, but love tidying.
Similar things could be said about my mentality when it comes to data
cleaning versus data tidying. Unfortunately, as in with life, one must
clean before one tidies. But let's start with some conceptual
definitions.

Cleaning refers to the process of scrubbing the data into a way that
makes sense to you, the analyst. Oftentimes, and especially in public
sector organizations, the data is not clean. Whether you are looking at
the output of a SurveyMonkey survey or a canned report that is run from
the IT department, your data will come in all shapes and sizes.

Here is the first major departure from what you may have been taught
about data analysis in Excel. When you get messy data \emph{do not}
change individual cell values (if you can at all help it). Recall from
the introductory chapter the difference between cells and fields.
Fields, as a reminder, are columns that represent one variable. Whenever
possible, use data analysis tools to make changes to the entire field,
rather than specific cells. Most data analysis software, outside of
Excel, make it difficult or impossible to change individual cell values.
This is important for several reasons, most of which we will get to in
the next chapter on reproducibility. But for now, thinking in terms of
fields, and making changes to entire fields, will save you \emph{a lot}
of work and headache in the long run. Let's look at a sample dataset
that may be similar to one you would encounter in real life. Here is a
survey collected by a field manager of a local parks and recreation
department on employment.

\begin{Shaded}
\begin{Highlighting}[]
\KeywordTok{library}\NormalTok{(tidyverse)}

\NormalTok{sites <-}\StringTok{ }\KeywordTok{tribble}\NormalTok{(}
  \OperatorTok{~}\StringTok{"Employee"}\NormalTok{, }\OperatorTok{~}\StringTok{"Location"}\NormalTok{, }\OperatorTok{~}\StringTok{"Telecommute?"}\NormalTok{, }\OperatorTok{~}\StringTok{"Hire Date"}\NormalTok{,}
  \StringTok{"ron swanson"}\NormalTok{, }\StringTok{"Pawnee City Hall"}\NormalTok{, }\StringTok{"never"}\NormalTok{, }\StringTok{"Unknown"}\NormalTok{,}
  \StringTok{"Knope, Leslie"}\NormalTok{, }\StringTok{"Field Duty"}\NormalTok{, }\StringTok{"1 day/week"}\NormalTok{, }\StringTok{"2011-6-1"}\NormalTok{,}
  \StringTok{"Andy Dwyer"}\NormalTok{, }\StringTok{"sullivan street pit"}\NormalTok{, }\StringTok{"40 hours"}\NormalTok{, }\StringTok{"March 1, 2013"}\NormalTok{,}
  \StringTok{"Jerry Gergich"}\NormalTok{, }\StringTok{"City Hall"}\NormalTok{, }\StringTok{"never"}\NormalTok{, }\StringTok{"6/1/1985"}\NormalTok{,}
  \StringTok{"Garry Gergich"}\NormalTok{, }\StringTok{"City Hall"}\NormalTok{, }\StringTok{"never"}\NormalTok{, }\StringTok{"6/1/1985"}\NormalTok{,}
  \StringTok{"ben wyatt"}\NormalTok{, }\StringTok{"Partridge, Minnesota"}\NormalTok{,}\StringTok{""}\NormalTok{ , }\StringTok{"Jan. 1, 2010"}
\NormalTok{)}

\NormalTok{sites }\OperatorTok\StringTok{ }\KeywordTok{datatable}\NormalTok{(}
    \DataTypeTok{extensions =} \StringTok{'Buttons'}\NormalTok{, }
    \DataTypeTok{options =} \KeywordTok{list}\NormalTok{(}\DataTypeTok{dom =} \StringTok{'Bfrtip'}\NormalTok{, }
                   \DataTypeTok{buttons =} \StringTok{'excel'}\NormalTok{,}
                   \DataTypeTok{searching =} \OtherTok{FALSE}\NormalTok{))}
\end{Highlighting}
\end{Shaded}

\hypertarget{htmlwidget-098b1d77ab6e7c33511b}{}

In this example, it would be trivial to go in to the Excel file and
clean up the dates, names, and locations by hand. However, you could
imagine this survey replicated for a department of forty employees. It
quickly becomes unfeasable to make those edits by hand. When this is the
case, there are functions in Excel and R that will make your life much
easier.

Here is annotated code for how I would go about cleaning this table in
R. The friendly syntax of the \texttt{tidyverse} packages makes it easy
to follow along, even if you aren't comfortable writing it yourself.

\begin{Shaded}
\begin{Highlighting}[]
\NormalTok{sites_cleaned <-}\StringTok{ }\NormalTok{sites }\OperatorTok\StringTok{ }\CommentTok{#creating a new table called "sites_cleaned", starting with the old table "sites"}
\StringTok{  }\KeywordTok{mutate}\NormalTok{(}\DataTypeTok{Employee =} \KeywordTok{if_else}\NormalTok{(Employee }\OperatorTok{==}\StringTok{ "Knope, Leslie"}\NormalTok{, }\StringTok{"Leslie Knope"}\NormalTok{, Employee)) }\OperatorTok\StringTok{ }
\StringTok{  }\KeywordTok{separate}\NormalTok{(Employee, }\DataTypeTok{into =} \KeywordTok{c}\NormalTok{(}\StringTok{"first_name"}\NormalTok{, }\StringTok{"last_name"}\NormalTok{)) }\OperatorTok\StringTok{ }
\StringTok{  }\KeywordTok{rename}\NormalTok{(}\DataTypeTok{location =}\NormalTok{ Location,}
         \DataTypeTok{telecommute_hours =}\StringTok{`}\DataTypeTok{Telecommute?}\StringTok{`}\NormalTok{,}
         \DataTypeTok{hire_date =} \StringTok{`}\DataTypeTok{Hire Date}\StringTok{`}\NormalTok{) }\OperatorTok\StringTok{ }
\StringTok{  }\KeywordTok{mutate}\NormalTok{(}\DataTypeTok{first_name =} \KeywordTok{str_to_title}\NormalTok{(first_name),}
         \DataTypeTok{last_name =} \KeywordTok{str_to_title}\NormalTok{(last_name),}
         \DataTypeTok{location =} \KeywordTok{str_to_title}\NormalTok{(location)) }\OperatorTok\StringTok{ }
\StringTok{  }\KeywordTok{mutate}\NormalTok{(}\DataTypeTok{location =} \KeywordTok{case_when}\NormalTok{(}
           \KeywordTok{str_detect}\NormalTok{(location, }\StringTok{"City Hall"}\NormalTok{) }\OperatorTok{~}\StringTok{ "In Office"}\NormalTok{,}
           \KeywordTok{str_detect}\NormalTok{(location, }\StringTok{"Field"}\NormalTok{) }\OperatorTok{~}\StringTok{ "In Field"}\NormalTok{,}
           \KeywordTok{str_detect}\NormalTok{(location, }\StringTok{"Street"}\NormalTok{) }\OperatorTok{~}\StringTok{ "In Field"}\NormalTok{,}
           \OtherTok{TRUE} \OperatorTok{~}\StringTok{ "Other"}\NormalTok{),}
         \DataTypeTok{telecommute_hours =} \KeywordTok{case_when}\NormalTok{(}
\NormalTok{           telecommute_hours }\OperatorTok{==}\StringTok{ "never"} \OperatorTok{~}\StringTok{ }\DecValTok{0}\NormalTok{,}
\NormalTok{           telecommute_hours }\OperatorTok{==}\StringTok{ "1 day/week"} \OperatorTok{~}\StringTok{ }\DecValTok{8}\NormalTok{,}
\NormalTok{           telecommute_hours }\OperatorTok{==}\StringTok{ "40 hours"} \OperatorTok{~}\StringTok{ }\DecValTok{40}
\NormalTok{         )}
\NormalTok{         )}
\end{Highlighting}
\end{Shaded}

\section{Thinking in Pivot Tables-- From Wide to
Long.}\label{thinking-in-pivot-tables-from-wide-to-long.}

Pivot tables are amazing. THey are the world's most common, most
helpful, and most underrated data analysis tool. PowerBI interactive
charts and graphs are just pivot tables in disguise. Understanding what
is needed to make a pivot table work is the key to the wide world of
data analysis.

A pivot table groups data by field and allows the user to drag fields to
the rows or columns of the pivot table. This is effective when each
field is a variable (something that can change), and each row is a
seperate observation of some phenomena of interest.

In short, pivot tables depend on \textbf{tidy data}.

Tidy data is the way your data should be organized before you begin your
analysis. In tidy data, each column is a \emph{variable}, each row is an
\emph{observation}, and each table is an \emph{associated set of
observations}. What does that mean in practice? Consider the following
example.

Below is a table\footnote{Data was created for demonstration purposes}
that shows types of retirement visits for a month at a state's
Department of Retirement Services by the employee who took the visit and
the visit type.

\begin{Shaded}
\begin{Highlighting}[]
\NormalTok{visits <-}\StringTok{ }\KeywordTok{tribble}\NormalTok{(}
  \OperatorTok{~}\StringTok{"Employee"}\NormalTok{, }\OperatorTok{~}\StringTok{"Phone Visits"}\NormalTok{, }\OperatorTok{~}\StringTok{"Office Visits"}\NormalTok{, }\OperatorTok{~}\StringTok{"Online Visits"}\NormalTok{,}
  \StringTok{"Danielle"}\NormalTok{, }\DecValTok{6}\NormalTok{, }\DecValTok{11}\NormalTok{, }\DecValTok{23}\NormalTok{,}
  \StringTok{"Ramona"}\NormalTok{, }\DecValTok{11}\NormalTok{, }\DecValTok{5}\NormalTok{, }\DecValTok{18}\NormalTok{,}
  \StringTok{"Ross"}\NormalTok{, }\DecValTok{10}\NormalTok{, }\DecValTok{10}\NormalTok{, }\DecValTok{10} 
\NormalTok{)}

\NormalTok{knitr}\OperatorTok{::}\KeywordTok{kable}\NormalTok{(visits, }\DataTypeTok{caption =} \StringTok{"Visits to the Dept. of Retirement Services in a given month"}\NormalTok{)}
\end{Highlighting}
\end{Shaded}

\begin{table}

\caption{\label{tab:tables-visits}Visits to the Dept. of Retirement Services in a given month}
\centering
\begin{tabular}[t]{l|r|r|r}
\hline
Employee & Phone Visits & Office Visits & Online Visits\\
\hline
Danielle & 6 & 11 & 23\\
\hline
Ramona & 11 & 5 & 18\\
\hline
Ross & 10 & 10 & 10\\
\hline
\end{tabular}
\end{table}

Data are frequently displayed in this ``wide'' format. It works great
for presentation, but not great for data analysis.

The shortcomings of data in this format may become apparent when you
attempt to work with the data in a pivot table. This is becuase our
columns aren't truly variables. You can drag the fields from the top row
to the grey box below, for columns, and the left, for rows. This becomes
unmanegable quickly.

\begin{Shaded}
\begin{Highlighting}[]
\NormalTok{rpivotTable}\OperatorTok{::}\KeywordTok{rpivotTable}\NormalTok{(visits, }\DataTypeTok{width =} \StringTok{"60%"}\NormalTok{, }\DataTypeTok{height =} \StringTok{"60%"}\NormalTok{)}
\end{Highlighting}
\end{Shaded}

\hypertarget{htmlwidget-dc0275d8158601af18ad}{}

Let's apply our criteria of tidy data to this set:

\begin{itemize}
\tightlist
\item
  Variables

  \begin{itemize}
  \tightlist
  \item
    At first glance, it doensn't look like this is a problem. But think
    again. Is \texttt{phone\ visits} really a variable? Or is the real
    variable of interest number of visits? And are our column names
    actually variables too (type of visit)?
  \end{itemize}
\end{itemize}

Let's take another swing at setting up our table for data analysis
purposes. This can be accomplished easily in R using the code below, or
in Excel by loading the data with \texttt{Get\ and\ Transform}
-\textgreater{} selecting the three ``visits'' columns -\textgreater{}
right clicking -\textgreater{} and selecting ``unpivot columns.''

\begin{Shaded}
\begin{Highlighting}[]
\CommentTok{#We have already loaded the "tidyverse" library so we do not have to do it again}

\NormalTok{pivot_visits <-}\StringTok{ }\NormalTok{visits }\OperatorTok\StringTok{ }\CommentTok{#we are editing the "visits" table already created by storing it in a new table pivot_visits}
\StringTok{  }\KeywordTok{pivot_longer}\NormalTok{(}\OperatorTok{-}\NormalTok{Employee, }\DataTypeTok{names_to =} \StringTok{"Visit Type"}\NormalTok{, }\DataTypeTok{values_to =} \StringTok{"Number of Visits"}\NormalTok{) }\CommentTok{#using pivot_longer on every column except "employee" and setting the name of the new columns}


\NormalTok{knitr}\OperatorTok{::}\KeywordTok{kable}\NormalTok{(pivot_visits, }\DataTypeTok{caption =} \StringTok{"Visits to the Dept. of Retirement Services in a given month"}\NormalTok{)}
\end{Highlighting}
\end{Shaded}

\begin{table}

\caption{\label{tab:unnamed-chunk-4}Visits to the Dept. of Retirement Services in a given month}
\centering
\begin{tabular}[t]{l|l|r}
\hline
Employee & Visit Type & Number of Visits\\
\hline
Danielle & Phone Visits & 6\\
\hline
Danielle & Office Visits & 11\\
\hline
Danielle & Online Visits & 23\\
\hline
Ramona & Phone Visits & 11\\
\hline
Ramona & Office Visits & 5\\
\hline
Ramona & Online Visits & 18\\
\hline
Ross & Phone Visits & 10\\
\hline
Ross & Office Visits & 10\\
\hline
Ross & Online Visits & 10\\
\hline
\end{tabular}
\end{table}

Now this is a table that is much easier to analyze in an Excel pivot
table or with a variety of R functions. Using data in this format, it is
easy to recreate the original table for presentation, while also giving
a variety of options for formatting and plotting. Use the pivot table
below to recreate the original table using the tidy data. *Hint- Instead
of Count, select Sum -\textgreater{} Number of Visits as the value
field. It is far easier to work with fields when they are in a tidy
format.

\begin{Shaded}
\begin{Highlighting}[]
\NormalTok{rpivotTable}\OperatorTok{::}\KeywordTok{rpivotTable}\NormalTok{(pivot_visits, }\DataTypeTok{width =} \StringTok{"60%"}\NormalTok{, }\DataTypeTok{height =} \StringTok{"400px"}\NormalTok{)}
\end{Highlighting}
\end{Shaded}

\hypertarget{htmlwidget-b92334b9d9e93c3857d1}{}

When we get to the next chapter, you will learn several alternatives to
pivot tables that use the same principles, but are more reproducible.

\section{Using lower level data}\label{using-lower-level-data}

Let's introduce a slightly more complicated tidy data problem, using the
same base data as before.

\begin{Shaded}
\begin{Highlighting}[]
\NormalTok{visits_retirements <-}\StringTok{ }\KeywordTok{tribble}\NormalTok{(}
  \OperatorTok{~}\StringTok{"Employee"}\NormalTok{, }\OperatorTok{~}\StringTok{"Phone Visits"}\NormalTok{, }\OperatorTok{~}\StringTok{"Phone Retirements"}\NormalTok{, }\OperatorTok{~}\StringTok{"Office Visits"}\NormalTok{, }\OperatorTok{~}\StringTok{"Office Retirements"}\NormalTok{, }\OperatorTok{~}\StringTok{"Online Visits"}\NormalTok{, }\OperatorTok{~}\StringTok{"Online Retirements"}\NormalTok{,}
  \StringTok{"Danielle"}\NormalTok{, }\DecValTok{6}\NormalTok{, }\DecValTok{4}\NormalTok{, }\DecValTok{11}\NormalTok{, }\DecValTok{8}\NormalTok{, }\DecValTok{23}\NormalTok{, }\DecValTok{15}\NormalTok{,}
  \StringTok{"Ramona"}\NormalTok{, }\DecValTok{11}\NormalTok{, }\DecValTok{7}\NormalTok{, }\DecValTok{5}\NormalTok{, }\DecValTok{3}\NormalTok{, }\DecValTok{18}\NormalTok{, }\DecValTok{15}\NormalTok{,}
  \StringTok{"Ross"}\NormalTok{, }\DecValTok{10}\NormalTok{, }\DecValTok{8}\NormalTok{, }\DecValTok{10}\NormalTok{, }\DecValTok{7}\NormalTok{, }\DecValTok{10}\NormalTok{, }\DecValTok{9} 
\NormalTok{)}

\NormalTok{knitr}\OperatorTok{::}\KeywordTok{kable}\NormalTok{(visits_retirements, }\DataTypeTok{caption =} \StringTok{"Visits to the Dept. of Retirement Services in a given month by employee and associated client retirements"}\NormalTok{)}
\end{Highlighting}
\end{Shaded}

\begin{table}

\caption{\label{tab:unnamed-chunk-6}Visits to the Dept. of Retirement Services in a given month by employee and associated client retirements}
\centering
\begin{tabular}[t]{l|r|r|r|r|r|r}
\hline
Employee & Phone Visits & Phone Retirements & Office Visits & Office Retirements & Online Visits & Online Retirements\\
\hline
Danielle & 6 & 4 & 11 & 8 & 23 & 15\\
\hline
Ramona & 11 & 7 & 5 & 3 & 18 & 15\\
\hline
Ross & 10 & 8 & 10 & 7 & 10 & 9\\
\hline
\end{tabular}
\end{table}

Hopefully you will see a similar pattern here. Now, there are three
variables: Visit type, number of visits, and number of retirements.
Again, this data works fine for presentation but could use tidying to
ease in analysis.

\begin{Shaded}
\begin{Highlighting}[]
\NormalTok{visits_retirements }\OperatorTok
\StringTok{  }\NormalTok{DT}\OperatorTok{::}\KeywordTok{datatable}\NormalTok{(}
    \DataTypeTok{extensions =} \StringTok{'Buttons'}\NormalTok{, }
    \DataTypeTok{options =} \KeywordTok{list}\NormalTok{(}\DataTypeTok{dom =} \StringTok{'Bfrtip'}\NormalTok{, }
                   \DataTypeTok{buttons =} \StringTok{'excel'}\NormalTok{,}
                   \DataTypeTok{searching =} \OtherTok{FALSE}\NormalTok{))}
\end{Highlighting}
\end{Shaded}

\hypertarget{htmlwidget-73af6fa0012bdaaa37da}{}

Try to tidy this in R or Excel Get and Transform. See this
footnote\footnote{powerquery hints} or look at the code if you need a
hint.

\begin{Shaded}
\begin{Highlighting}[]
\NormalTok{visits_retirements_tidy <-}\StringTok{ }\NormalTok{visits_retirements }\OperatorTok
\StringTok{  }\KeywordTok{pivot_longer}\NormalTok{(}\DataTypeTok{cols =} \OperatorTok{-}\NormalTok{Employee, }
               \DataTypeTok{names_to =} \KeywordTok{c}\NormalTok{(}\StringTok{"Visit Location"}\NormalTok{, }\StringTok{"Type"}\NormalTok{), }
               \DataTypeTok{names_sep =} \StringTok{" "}\NormalTok{)}

\KeywordTok{print}\NormalTok{(visits_retirements_tidy)}
\end{Highlighting}
\end{Shaded}

\begin{verbatim}
## # A tibble: 18 x 4
##    Employee `Visit Location` Type        value
##    <chr>    <chr>            <chr>       <dbl>
##  1 Danielle Phone            Visits          6
##  2 Danielle Phone            Retirements     4
##  3 Danielle Office           Visits         11
##  4 Danielle Office           Retirements     8
##  5 Danielle Online           Visits         23
##  6 Danielle Online           Retirements    15
##  7 Ramona   Phone            Visits         11
##  8 Ramona   Phone            Retirements     7
##  9 Ramona   Office           Visits          5
## 10 Ramona   Office           Retirements     3
## 11 Ramona   Online           Visits         18
## 12 Ramona   Online           Retirements    15
## 13 Ross     Phone            Visits         10
## 14 Ross     Phone            Retirements     8
## 15 Ross     Office           Visits         10
## 16 Ross     Office           Retirements     7
## 17 Ross     Online           Visits         10
## 18 Ross     Online           Retirements     9
\end{verbatim}

In this case, we actually pivoted too far. It will probably be more
useful to have the counts of visits and retirements in their own
category. Keep in mind the scope of the observation-- It is perfectly
valid for each to have their own column, as it is visits and retirements
per month.

\begin{Shaded}
\begin{Highlighting}[]
\NormalTok{visits_retirements_tidy2 <-}\StringTok{ }\NormalTok{visits_retirements_tidy }\OperatorTok\StringTok{ }
\StringTok{  }\KeywordTok{pivot_wider}\NormalTok{(}\DataTypeTok{id_cols =} \KeywordTok{c}\NormalTok{(Employee, }\StringTok{`}\DataTypeTok{Visit Location}\StringTok{`}\NormalTok{, Type), }\DataTypeTok{names_from =}\NormalTok{ Type, }\DataTypeTok{values_from =}\NormalTok{ value)}

\KeywordTok{print}\NormalTok{(visits_retirements_tidy2)}
\end{Highlighting}
\end{Shaded}

\begin{verbatim}
## # A tibble: 9 x 4
##   Employee `Visit Location` Visits Retirements
##   <chr>    <chr>             <dbl>       <dbl>
## 1 Danielle Phone                 6           4
## 2 Danielle Office               11           8
## 3 Danielle Online               23          15
## 4 Ramona   Phone                11           7
## 5 Ramona   Office                5           3
## 6 Ramona   Online               18          15
## 7 Ross     Phone                10           8
## 8 Ross     Office               10           7
## 9 Ross     Online               10           9
\end{verbatim}

From here, it is easy to do calculations based on fields, rather than
cells. For example, in R or Get and Transform, you could add the
following:

\begin{Shaded}
\begin{Highlighting}[]
\NormalTok{visits_pct <-}\StringTok{ }\NormalTok{visits_retirements_tidy2 }\OperatorTok\StringTok{ }
\StringTok{  }\KeywordTok{mutate}\NormalTok{(}\DataTypeTok{pct_retirements =}\NormalTok{ Retirements }\OperatorTok{/}\StringTok{ }\NormalTok{Visits)}

\KeywordTok{print}\NormalTok{(visits_pct)}
\end{Highlighting}
\end{Shaded}

\begin{verbatim}
## # A tibble: 9 x 5
##   Employee `Visit Location` Visits Retirements pct_retirements
##   <chr>    <chr>             <dbl>       <dbl>           <dbl>
## 1 Danielle Phone                 6           4           0.667
## 2 Danielle Office               11           8           0.727
## 3 Danielle Online               23          15           0.652
## 4 Ramona   Phone                11           7           0.636
## 5 Ramona   Office                5           3           0.6  
## 6 Ramona   Online               18          15           0.833
## 7 Ross     Phone                10           8           0.8  
## 8 Ross     Office               10           7           0.7  
## 9 Ross     Online               10           9           0.9
\end{verbatim}

And then, one of the most useful things you can do is develop formulas
by grouping of rows. For example, you may want to know the total number
of visits and retirements by retiree, regardless of visit location. That
can be accomplished in a pivot table.

\section{So how is this democratic?}\label{so-how-is-this-democratic}

\section{Practice problems}\label{practice-problems}

\chapter{Reproducible Analysis}\label{reproducible-analysis}

The second principle of democratic data analysis is reproducability. By
this, I mean anything that makes it easy for someone else (or you,
several months from now), to look at your analysis and understand what
is going on. This is where classic data analysis in Excel falls short. I
believe it is almost a universal experience in the public sector to
recieve a workbook full of broken links, formulas pointing in every
direction, and no sense of where the original data is.

In thinking about creating reproducible data analysis, it is important
to keep in mind that data analasys should be structured from beginning
to end, like a story. In the beginning, there is raw data that you
pulled from a report, compiled yourself, or otherwise recieved. In Act
1, you use the practices we learned in the previous section to make the
raw data tidy-- Without distroying the original data. In Act 2, which
will be the next chapter, you use your data to create a picture of the
world before you share it with others in the final Act 3.

The practices of reproducability that you will use here apply throughout
the other chapters. It may seem like a waste of time, but if you have
ever come back to a complicated excel workbook after spending even days
away, this will make your life much easier.

\section{Comment Comment Comment}\label{comment-comment-comment}

\chapter{Data Modeling}\label{data-modeling}

\section{Why Model?}\label{why-model}

Models transform data into decision making. \#\# Example two

\section{Assumptions}\label{assumptions}

\chapter{Visualization}\label{visualization}

\section{Show and Tell}\label{show-and-tell}

\section{Vizualizaiton is anything that presents your evidence-- think
critically about
it!}\label{vizualizaiton-is-anything-that-presents-your-evidence-think-critically-about-it}

\chapter{Applications}\label{applications}

\section{Tying it all together}\label{tying-it-all-together}

\chapter{Resources}\label{resources}

\bibliography{book.bib,packages.bib}

\end{document}
