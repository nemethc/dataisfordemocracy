\documentclass[]{book}
\usepackage{lmodern}
\usepackage{amssymb,amsmath}
\usepackage{ifxetex,ifluatex}
\usepackage{fixltx2e} % provides \textsubscript
\ifnum 0\ifxetex 1\fi\ifluatex 1\fi=0 % if pdftex
  \usepackage[T1]{fontenc}
  \usepackage[utf8]{inputenc}
\else % if luatex or xelatex
  \ifxetex
    \usepackage{mathspec}
  \else
    \usepackage{fontspec}
  \fi
  \defaultfontfeatures{Ligatures=TeX,Scale=MatchLowercase}
\fi
% use upquote if available, for straight quotes in verbatim environments
\IfFileExists{upquote.sty}{\usepackage{upquote}}{}
% use microtype if available
\IfFileExists{microtype.sty}{%
\usepackage[]{microtype}
\UseMicrotypeSet[protrusion]{basicmath} % disable protrusion for tt fonts
}{}
\PassOptionsToPackage{hyphens}{url} % url is loaded by hyperref
\usepackage[unicode=true]{hyperref}
\hypersetup{
            pdftitle={Doing Democratic Data Analysis},
            pdfauthor={Corban Nemeth},
            pdfborder={0 0 0},
            breaklinks=true}
\urlstyle{same}  % don't use monospace font for urls
\usepackage{natbib}
\bibliographystyle{apalike}
\usepackage{longtable,booktabs}
% Fix footnotes in tables (requires footnote package)
\IfFileExists{footnote.sty}{\usepackage{footnote}\makesavenoteenv{long table}}{}
\usepackage{graphicx,grffile}
\makeatletter
\def\maxwidth{\ifdim\Gin@nat@width>\linewidth\linewidth\else\Gin@nat@width\fi}
\def\maxheight{\ifdim\Gin@nat@height>\textheight\textheight\else\Gin@nat@height\fi}
\makeatother
% Scale images if necessary, so that they will not overflow the page
% margins by default, and it is still possible to overwrite the defaults
% using explicit options in \includegraphics[width, height, ...]{}
\setkeys{Gin}{width=\maxwidth,height=\maxheight,keepaspectratio}
\IfFileExists{parskip.sty}{%
\usepackage{parskip}
}{% else
\setlength{\parindent}{0pt}
\setlength{\parskip}{6pt plus 2pt minus 1pt}
}
\setlength{\emergencystretch}{3em}  % prevent overfull lines
\providecommand{\tightlist}{%
  \setlength{\itemsep}{0pt}\setlength{\parskip}{0pt}}
\setcounter{secnumdepth}{5}
% Redefines (sub)paragraphs to behave more like sections
\ifx\paragraph\undefined\else
\let\oldparagraph\paragraph
\renewcommand{\paragraph}[1]{\oldparagraph{#1}\mbox{}}
\fi
\ifx\subparagraph\undefined\else
\let\oldsubparagraph\subparagraph
\renewcommand{\subparagraph}[1]{\oldsubparagraph{#1}\mbox{}}
\fi

% set default figure placement to htbp
\makeatletter
\def\fps@figure{htbp}
\makeatother

\usepackage{booktabs}

\title{Doing Democratic Data Analysis}
\author{Corban Nemeth}
\date{2020-04-06}

\begin{document}
\maketitle

{
\setcounter{tocdepth}{1}
\tableofcontents
}
\chapter*{Preface}\label{preface}
\addcontentsline{toc}{chapter}{Preface}

I believe that data, \emph{in the hands of public administrators and
policy analysts}\footnote{\emph{Not IT departments}}, has the power to
transform the way government works.

Big questions will, and should, be asked of big data--- the role of
government in regulating algorithmic bias, facial recognition, and
consumer data privacy is a vital conversation. However, these topics
should not detract or deter public administrators and policy analysts
from leaning into \textbf{small data} for decision-making purposes.

Public administrators and analysts who are data literate will be able to
make and inform better decisions while avoiding the pitfalls posed by
the latest technological trends. This book represents an opportunity for
public administrators and policy analysts to join their subject matter
expertise with foundation principles and practices of democratic data
analysis--- data analysis that is \textbf{transparent, relevant, and
grounded in the context of ethical and effective governance}.

\section*{Who is this guide for?}\label{who-is-this-guide-for}
\addcontentsline{toc}{section}{Who is this guide for?}

This guide is for:

\begin{itemize}
\tightlist
\item
  the budget analyst at the Department of Fish and Wildlife who has to
  compile a monthly report analyzing revenues,
\item
  the manager at the Department of Social and Health Services who is
  tracking inventory, and
\item
  the research analyst working for the state Legislature who wants to
  incorporate data into her work session on the latest policy debate.
\end{itemize}

\section*{What will I learn?}\label{what-will-i-learn}
\addcontentsline{toc}{section}{What will I learn?}

You will learn an opinionated framework for data analysis in public
sector organizations. By opinionated, I mean that I will teach you what
I think is the right way to do things given my own experience as a
public sector policy and data analyst. Your experience might differ--
and that's great. I hope that where you can use your experience in place
of mine, you do to the fullest extent. With that in mind, it is often
said that you have to know the rules to break them, so I will teach you
the ``rules'' as I understand them.

\chapter{Introduction}\label{intro}

\section{What is democratic data
analysis?}\label{what-is-democratic-data-analysis}

This is little-d democracy here, people. Not a guide to data analysis
for Democrats, but an approach to data that empowers the subject matter
experts within public organizations to do their best work.

\section{Do I need to be an Excel god before I
begin?}\label{do-i-need-to-be-an-excel-god-before-i-begin}

\textbf{No.}

\section{What tools should I use?}\label{what-tools-should-i-use}

This guide is written to be technology-agnostic. The changing nature of
technology and the \emph{ahem} lag of state agencies to pick up tools
means that the principles outlined in this book supersede specific
technologies.

However, you should start somewhere. This guide will include examples
for both Excel and \texttt{R}. Government runs on Excel, so all of the
examples and exercises will be Excel compatible. If you are comfortable
with Excel\footnote{aka you use \texttt{vlookup}, \texttt{index(match)},
  pivot tables, or \emph{Get \& Transform} on a somewhat regular basis}
and want to challenge yourself, boost your resume, and become a data
wizard\footnote{For example, I used R to create this entire website}, I
would highly recommend learning \texttt{R}. This guide will walk you
through all the steps necessary to learn how to analyze data
democratically using both tools, but focusing on the principles and
practices that are vital regardless of technology.

\section{Getting Started with Excel}\label{getting-started-with-excel}

All examples

\section{Getting Started with R}\label{getting-started-with-r}

I would \textbf{HIGHLY} recommend you take an afternoon to work through
Chapters 1, 2, and 3 of
``\href{https://rstudio-conf-2020.github.io/r-for-excel/index.html}{\emph{R
for Excel Users}}''. You will thank me later.

\section{How to Think about Democratic Data
Analysis}\label{how-to-think-about-democratic-data-analysis}

What is data analysis? It may be easier to start with what data analysis
isn't.

\begin{enumerate}
\def\labelenumi{\arabic{enumi}.}
\tightlist
\item
  Data analysis isn't math in Excel or a set of specific calculations.
\end{enumerate}

\begin{itemize}
\tightlist
\item
  Calculations are great, but \texttt{a7\ +\ b8} in Excel is
  deterministic. It gives you one answer. This book is not interested in
  data analysis that gives you the right answer, because there is no
  such thing. There are many answers to many questions, depending on how
  those questions are asked and how the data is analyzed.
\end{itemize}

\begin{enumerate}
\def\labelenumi{\arabic{enumi}.}
\setcounter{enumi}{1}
\tightlist
\item
  Data analysis isn't statistics.
\end{enumerate}

\begin{itemize}
\tightlist
\item
  Look, regressions are great, but that isn't the crux of this book.
  This book is about reading and telling the story of your data in a way
  that can complement expertise and experience to make better decisions.
  Statistics are often used as a cheap stand-in for domain expertise and
  are often abused in favor of trusting the analyst or administrator to
  back up their assumptions with both quantitative and qualitiative
  data.
\end{itemize}

\begin{enumerate}
\def\labelenumi{\arabic{enumi}.}
\setcounter{enumi}{2}
\tightlist
\item
  Data analysis isn't research methods.
\end{enumerate}

\begin{itemize}
\tightlist
\item
  No set of tools and practices can stand in for asking the right
  questions, and transforming data into information to answer that
  question. This book will give you the tools to work with your
  quantitative data to answer relevant questions, but all good analysis
  begins with a good question.
\end{itemize}

So what are the principles of democratic data analysis?

\begin{enumerate}
\def\labelenumi{\arabic{enumi}.}
\tightlist
\item
  Democratic data analysis gives power to the subject matter expert.
\end{enumerate}

\begin{itemize}
\tightlist
\item
  Working with your own data lets you answer your question, rather than
  outsource those questions to IT departments who don't have the context
  necessary to turn data into information.
\end{itemize}

\begin{enumerate}
\def\labelenumi{\arabic{enumi}.}
\setcounter{enumi}{1}
\tightlist
\item
  Democratic data analysis is easily shared and reproduced.
\item
  Democratic data analysis is honest about its assumptions and
  limitations.
\item
  Democratic data analysis is structured to be easily understood and
  communicated.
\item
  Democratic data analysis is approached methodically.
\end{enumerate}

\section{Alternate Forms of Data
Analysis}\label{alternate-forms-of-data-analysis}

I want to contrast democratic data analysis with alternate forms of
analysis that I have seen in my time working in the public sector.

\subsection{Oligarchic Data Analysis}\label{oligarchic-data-analysis}

\begin{itemize}
\tightlist
\item
  small group of ``data people''
\end{itemize}

\subsection{Monarchic Data Analysis}\label{monarchic-data-analysis}

\begin{itemize}
\tightlist
\item
  Spreadsheets created decades ago, patched as necessary, and owned by a
  single or small group of people. Very unnaproachable. Can't ask
  questions.
\end{itemize}

\subsection{Dictitorial Data Analysis}\label{dictitorial-data-analysis}

*Not reproducible, all hard-coded, can't ask questions,

\subsection{Technocratic Data
Analysis}\label{technocratic-data-analysis}

*Excel workbooks straight from the depths of hell. Formulas on formulas
on macros.

\section{Principles of Democratic Data
Analysis}\label{principles-of-democratic-data-analysis}

\begin{itemize}
\tightlist
\item
  Think in terms of fields, not values
\item
  Leave breadcrumbs for others (and your future self)
\item
  Create a data pipeline and DO NOT DESTROY UNDERLYING DATA-- build
\item
  Make assumptions, and document them!
\item
  Show AND tell your results
\end{itemize}

\chapter{Tidy Data}\label{tidy-data}

\section{Tidy Data, or Why you should clean your
room}\label{tidy-data-or-why-you-should-clean-your-room}

\section{Example}\label{example}

\section{Pivoting}\label{pivoting}

\chapter{Reproducible Analysis}\label{reproducible-analysis}

We describe our methods in this chapter.

\section{Make your life easier}\label{make-your-life-easier}

\chapter{Data Modeling}\label{data-modeling}

Some \emph{significant} applications are demonstrated in this chapter.

\section{Example one}\label{example-one}

\section{Example two}\label{example-two}

\section{Assumptions}\label{assumptions}

\chapter{Visualization}\label{visualization}

We have finished a nice book.

\chapter{Applications}\label{applications}

\chapter{Resources}\label{resources}

\bibliography{book.bib,packages.bib}

\end{document}
