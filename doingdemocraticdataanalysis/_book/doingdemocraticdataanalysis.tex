\documentclass[]{book}
\usepackage{lmodern}
\usepackage{amssymb,amsmath}
\usepackage{ifxetex,ifluatex}
\usepackage{fixltx2e} % provides \textsubscript
\ifnum 0\ifxetex 1\fi\ifluatex 1\fi=0 % if pdftex
  \usepackage[T1]{fontenc}
  \usepackage[utf8]{inputenc}
\else % if luatex or xelatex
  \ifxetex
    \usepackage{mathspec}
  \else
    \usepackage{fontspec}
  \fi
  \defaultfontfeatures{Ligatures=TeX,Scale=MatchLowercase}
\fi
% use upquote if available, for straight quotes in verbatim environments
\IfFileExists{upquote.sty}{\usepackage{upquote}}{}
% use microtype if available
\IfFileExists{microtype.sty}{%
\usepackage[]{microtype}
\UseMicrotypeSet[protrusion]{basicmath} % disable protrusion for tt fonts
}{}
\PassOptionsToPackage{hyphens}{url} % url is loaded by hyperref
\usepackage[unicode=true]{hyperref}
\hypersetup{
            pdftitle={Doing Democratic Data Analysis},
            pdfauthor={Corban Nemeth},
            pdfborder={0 0 0},
            breaklinks=true}
\urlstyle{same}  % don't use monospace font for urls
\usepackage{natbib}
\bibliographystyle{apalike}
\usepackage{longtable,booktabs}
% Fix footnotes in tables (requires footnote package)
\IfFileExists{footnote.sty}{\usepackage{footnote}\makesavenoteenv{long table}}{}
\usepackage{graphicx,grffile}
\makeatletter
\def\maxwidth{\ifdim\Gin@nat@width>\linewidth\linewidth\else\Gin@nat@width\fi}
\def\maxheight{\ifdim\Gin@nat@height>\textheight\textheight\else\Gin@nat@height\fi}
\makeatother
% Scale images if necessary, so that they will not overflow the page
% margins by default, and it is still possible to overwrite the defaults
% using explicit options in \includegraphics[width, height, ...]{}
\setkeys{Gin}{width=\maxwidth,height=\maxheight,keepaspectratio}
\IfFileExists{parskip.sty}{%
\usepackage{parskip}
}{% else
\setlength{\parindent}{0pt}
\setlength{\parskip}{6pt plus 2pt minus 1pt}
}
\setlength{\emergencystretch}{3em}  % prevent overfull lines
\providecommand{\tightlist}{%
  \setlength{\itemsep}{0pt}\setlength{\parskip}{0pt}}
\setcounter{secnumdepth}{5}
% Redefines (sub)paragraphs to behave more like sections
\ifx\paragraph\undefined\else
\let\oldparagraph\paragraph
\renewcommand{\paragraph}[1]{\oldparagraph{#1}\mbox{}}
\fi
\ifx\subparagraph\undefined\else
\let\oldsubparagraph\subparagraph
\renewcommand{\subparagraph}[1]{\oldsubparagraph{#1}\mbox{}}
\fi

% set default figure placement to htbp
\makeatletter
\def\fps@figure{htbp}
\makeatother

\usepackage{booktabs}

\title{Doing Democratic Data Analysis}
\author{Corban Nemeth}
\date{2020-04-02}

\begin{document}
\maketitle

{
\setcounter{tocdepth}{1}
\tableofcontents
}
\chapter*{Preface}\label{preface}
\addcontentsline{toc}{chapter}{Preface}

I believe that data, \emph{in the hands of public administrators and
policy analysts}\footnote{\emph{Not IT departments}}, has the power to
transform the way government works.

Big questions will, and should, be asked of big data--- the role of
government in regulating algorithmic bias, facial recognition, and
consumer data privacy is a vital conversation. However, these topics
should not detract or deter public administrators and policy analysts
from leaning into \textbf{small data} for decision-making purposes.

Public administrators and analysts who are data literate will be able to
make and inform better decisions while avoiding the pitfalls posed by
the latest technological trends. This book represents an opportunity for
public administrators and policy analysts to join their subject matter
expertise with foundational principles and practices of democratic data
analysis--- data analysis that is \textbf{transparent, relevant, and
grounded in the context of ethical and effective governance}.

\section*{Who is this guide for?}\label{who-is-this-guide-for}
\addcontentsline{toc}{section}{Who is this guide for?}

This guide is for:

\begin{itemize}
\tightlist
\item
  the budget analyst at the Department of Fish and Wildlife who has to
  compile a monthly report analyzing revenues,
\item
  the manager at the Department of Social and Health Services who is
  tracking inventory, and
\item
  the research analyst working for the state Legislature who wants to
  incorporate data into her work session on the latest policy debate.
\end{itemize}

\section*{What will I learn?}\label{what-will-i-learn}
\addcontentsline{toc}{section}{What will I learn?}

You will learn an opinionated apprach to data analysis in public sector
organizations. By opionionated, I mean that I will teach you what I
think is the right way to do things given my own experience as a public
sector policy and data analyst. Your experience might differ-- and
that's great. I hope that where you can use your experience in place of
mine, you do to the fullest extent. With that in mind, it is often said
that you have to know the rules to break them, so I will teach you the
``rules''" as I understand them.

\chapter{Introduction}\label{intro}

\section{What is democratic data
analysis?}\label{what-is-democratic-data-analysis}

This is little-d democracy here, people. Not a guide to data analysis
for Democrats\footnote{*but if campaign folks are listening, don't
  forget about
  \href{https://blog.oup.com/2018/09/trump-beat-adas-big-data/}{Wisconson}},
but an approach to data that empowers the subject matter experts within
public organizations to do their best work.

\section{Do I need to be an Excel god before I
begin?}\label{do-i-need-to-be-an-excel-god-before-i-begin}

\textbf{No.}

\section{What tools should I use?}\label{what-tools-should-i-use}

This guide is written to be technology-agnostic. The changing nature of
technology and the \emph{ahem} lag of state agencies to pick up tools
means that the principles outlined in this book superseed specific
technologoes.

However, you should start somewhere. This guide will include examples
for both Excel and \texttt{R}. Government runs on Excel, so all of the
examples and exercises will be Excel compatible. If you are comfortable
with Excel\footnote{aka you use \texttt{vlookup}, \texttt{index(match)},
  pivot tables, or \emph{Get \& Transform} on a somewhat regular basis}
and want to challenge yourself, boost your resume, and become a data
wizard\footnote{For example, I used R to create this entire website}, I
would highly recommend learning \texttt{R}. This guide will walk you
through all the steps necessary to learn how to analyze data
democratically using both tools, but focusing on the principles and
practices that are vital regardless of technology.

\section{Getting Started with Excel}\label{getting-started-with-excel}

All examples

\section{Getting Started with R}\label{getting-started-with-r}

I would \textbf{HIGHLY} recommend you take an afternoon to work through
Chapters 1, 2, and 3 of
``\href{https://rstudio-conf-2020.github.io/r-for-excel/index.html}{\emph{R
for Excel Users}}''. You will thank me later.

\chapter{Tidy Data}\label{tidy-data}

Here is a review of existing methods.

\chapter{Methods}\label{methods}

We describe our methods in this chapter.

\chapter{Applications}\label{applications}

Some \emph{significant} applications are demonstrated in this chapter.

\section{Example one}\label{example-one}

\section{Example two}\label{example-two}

\chapter{Final Words}\label{final-words}

We have finished a nice book.

\bibliography{book.bib,packages.bib}

\end{document}
